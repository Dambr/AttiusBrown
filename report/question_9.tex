% 9. Какой информацией о факторах внешней среды Вы располагаете (детерминированные, случайные, интервально неопределенные, активное противодействие «противника» или конкурента)?
К детерминированным факторам внешней среды отношу:
\begin{enumerate}
    \item количество одновременно взаимодействующих со средой операторов - оператор всегда один;
    \item количество элементов управления, через которое оператор взаимодействует со средой разработки - исключено взаимодействие одновременно с несколькими элементами управления;
    \item среда разработки осуществляет свою работу под управлением одной операционной системы.
\end{enumerate}

К случайным факторам внешней среды отношу:
\begin{enumerate}
    \item на какой следующий элемент управления будет осуществлено воздействие оператора;
    \item количество ресурсов вычислительной системы, доступных для работы:
    \begin{itemize}
        \item количество процессного времени и объем оперативной памяти влияет на быстродействие среды разработки;
        \item политика безопасности определяет характер взаимодействия с ресурсами операционной системы, например, может не выполняться операция записи в файл журнала.
    \end{itemize}
\end{enumerate}

Конкурентная борьбся моей проектируемой системы от существующих аналогов на рынке осуществляется по критериям:
\begin{enumerate}
    \item объем функционала в его среде разработке;
    \item стоимость поставляемого функционала;
    \item время формирования поставки.
\end{enumerate}