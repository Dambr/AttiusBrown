% 3. Какова цель исследования системы? Где она может быть использована?
Исследование плагинных систем производится с целью разработки правил построения инструментальных средств конфигурирования.

Разработанные правила должны позволить разработчикам инструментальных средств конфигурирования, выполненными в виде плагинов, строить свои решения так, чтобы минимизировать затраты на сопровождение проекта и максимизировать число возможных комплектаций с реализацией требований на систему.

Это позволит сократить объем невостребованного заказчиком в поставке функционала, что снизит стоимость поставляемого функционала, следовательно поднимет конкуретоспособность программного решения на рынке.

Использование плагинной системы в качетве среды разработки подразумевает ее использование для сокращения временных и стоимостных издержек на разработку ПО, а так же отслеживание и информирование о потенциальных неисправностях или уязвимостях разрабатываемого ПО.

Она может быть использована в каждом из процессов разработки ПО. При этом важно, чтобы для каждого из процессов она имела вид наиболее точно характеризующий ее назначение для решаемых в рамках процесса задач.