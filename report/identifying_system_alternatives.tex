% Выявление альтернатив системы
% 12. Какое множество альтернатив системы (дискретное, непрерывное, дискретно–непрерывное) Вы рассматриваете?
% 13. Назовите основные структуры и параметры, характеризующие рассматриваемое Вами множество альтернатив системы.
% 14. При каких ограничениях на значения непрерывных параметров системы решается задача?
Так какие же варианты? Первое, что следует рассмотреть, как предоставлять заказчику только потребный функционал. Самый очевидный способ реализовать схему 1 требование - 1 плагин. Но этот вариант при достаточно большом объеме требований приведет к неконтролируемому росту сущностей в проекте ПО, следовательно, неконтролируемый рост затрат на его разработку и поддержку. Чтобы их компенсировать производитель будет вынужден повышать цену на свой IT продукт, что снижает конкурентоспособность бизнеса.

Другой альтернативой является схема все требования - 1 плагин. В этом случае коэффициент простоя будет слишком велик, если заказчик заинтересован лишь в части возможностей всего продукта. И службы пост продажного обслуживания будут перегружены, что так же приводит к стоимости поставляемого решения и снижения конкурентоспособности бизнеса.

В своей диссертации я решаю задачу оптимальной декомпозиции функционала по плагинам с целью снижения коэффициента простоя и контроля стоимости разработки и поддержки проекта.

Здесь тоже есть варианты. Первый - декомпозиция по частоте вызова функций. Если функция вызывается редко, она может быть обособлена, чтобы не тянуть свои зависимости в поставку. Часто вызываемые функции напротив, образуют ядро приложения.

Альтернативный - по функциональному признаку. Обосабливать функции по классу решаемых задач. Тогда, если заказчику потребены функции из определенного класса задач, то ему количество функций для решения задач других классов в поставке будет меньше.