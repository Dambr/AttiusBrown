% Выявление альтернатив системы
% 12. Какое множество альтернатив системы (дискретное, непрерывное, дискретно–непрерывное) Вы рассматриваете?
% 13. Назовите основные структуры и параметры, характеризующие рассматриваемое Вами множество альтернатив системы.
% 14. При каких ограничениях на значения непрерывных параметров системы решается задача?
С целью уменьшения $K_{f}$ видится применение схемы <<1 требование - 1 файл исходного кода - 1 плагин>>. Но этот вариант при достаточно большом объеме требований приведет к неконтролируемому росту $V_{c}$ в проекте ПО, следовательно, неконтролируемый рост затрат на разработку и поддержку. Чтобы их компенсировать производитель будет вынужден повышать цену на свое программное средство, что снижает конкурентоспособность бизнеса.

Другой альтернативой является схема <<все требования - 1 плагин>>. В этом случае $K_{f}$ будет слишком велик если заказчик заинтересован лишь в части возможностей всего продукта. Службы постпродажного обслуживания будут перегружены, что так же приводит к стоимости поставляемого решения и снижения конкурентоспособности бизнеса.

В своей диссертации я решаю задачу оптимальной декомпозиции функционала по плагинам с целью снижения $K_{f}$ и контроля роста показателя $V_{c}$.