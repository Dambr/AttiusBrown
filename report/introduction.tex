В современном мире важную роль в жизни общества занимают информационные технологии. В частности информационные технологии охватывают программное обеспечение цифровых устройств, которые человек применяет в повседневной жизнедеятельности: компьютеры, телефоны, планшеты, автомобили, умные холодильники и т.д. Основным поставщиком программного обеспечения для вышеупомянутых устройств являются IT-компании. Их имена известны, Google, Amazon, Yandex, Vk Group и др. Эти компании производят IT продукт, выполняющий заданный объем функционала и закрывающий обозначенные потребности потребителя.

IT продукты бывают совершенно разные. По их принадлежности:
\begin{itemize}
    \item для гражданского применения;
    \item для военного применения;
    \item для корпоративного применения и т.д.
\end{itemize}
По сфере применения:
\begin{itemize}
    \item офисное ПО;
    \item графические редакторы;
    \item игры и др.
\end{itemize}
По уровню критичности, например в авиационной технике:
\begin{itemize}
    \item катастрофическое;
    \item аварийное;
    \item сложное;
    \item усложнение условий полета;
    \item без последствий.
\end{itemize}

В том числе есть класс программных решение - плагинные системы. Это способ организации приложения таким образом, что его конечный объем функционала характеризуется количеством установленных в него расширений - плагинов.

Каждый отдельный плагин включает в себя конечное множество функционала, который в свою очередь основан на требованиях. Именно характер реализованных требований, а так же их объем потребен заказчику. Все остальное - плагины, их взаимосвязи, язык программирования, на котором они реализованы, библиотеки, которые задействованы, примененные технологии - все это скрыто от заказчика и зачастую его не интересует. Заказчику интересно, какие свои бытовые или бизнес потребности он сможет закрыть от применения IT продукта.

На стадии проектирования приложения зачастую неизвестно, какие требования будут востребованы у заказчиков и определить структуру приложения невозможно. Кроме того, перечень наиболее востребованного функционала может изменяться с течением времени. При необходимости формирования поставки, включающей требуемый объем функционала, поставщик зачастую включает и тот функционал, который не востребован и не оплачен заказчиком, но без которого не быть поставлен востребованный. Это связано с существованием зависимостей у функционала друг на друга.

Моя диссертационная работа посвящена поиску оптимальной структуры приложения, которая бы с одной стороны позволяла формировать поставки с минимальным числом невостребованного у заказчика функционала, а с другой сдерживала неконтролируемый рост сущностей и, тем самым, сдерживала стоимость сопровождения проекта.