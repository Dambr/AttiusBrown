% 5. Из каких подсистем состоит система?
Плагинная система состоит из плагинов, состав которых определяет ее функциональные возможности.

Для использования плагинной системы как инструментальной среды разработки ПО считаю, что она должна включать такие плагины как:
\begin{itemize}
    \item плагин, предоставляющий редактор с подсветкой синтаксиса и возможностями автодополнения для использумого в разрабатываемом проекте языка программирования;
    \item плагин навигации и поиска компонетов в проекте разрабатываемого ПО;
    \item плагин управления составом подключаемых компиляторов и правилами сборки проекта разрабатываемого ПО;
    \item плагин взаимодействия с системой контроля версий (СКВ) и др.
\end{itemize}