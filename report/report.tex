\documentclass{article}
\usepackage{tempora}
\usepackage{hyphenat}
\usepackage{indentfirst}
\usepackage{tabularx}
\usepackage{caption}
\usepackage{graphicx}
\graphicspath{ {./images/} }
\usepackage{float}
\usepackage[english, russian]{babel}
\hyphenpenalty=10000
\exhyphenpenalty=10000

% Ответ должен быть из 2-3х абзацев по 3-4 предложения в каждом
% annakd16@mail.ru - подготовленный документ отправлять сюда
% Дьячук Анна Константиновна

% 1. Какую систему Вы изучаете в Вашей диссертационной работе? Что является объектом исследования.
% 2. Каково назначение Вашей системы?
% 3. Какова цель исследования системы? Где она может быть использована?
% 4. Частью какой надсистемы является изучаемая система?
% 5. Из каких подсистем состоит система?
% 6. Какие задачи решают подсистемы в составе Вашей системы?
% 7. Сформулируйте кратко сценарий функционирования системы.
% 8. Какие факторы внешней среды Вы учитываете при анализе функционирования системы?
% 9. Какой информацией о факторах внешней среды Вы располагаете (детерминированные, случайные, интервально неопределенные, активное противодействие «противника» или конкурента)?
% 10. Какие показатели эффективности системы в целом и ее подсистем Вы рассматриваете?
% 11. Как в этих показателях учитывается информация о внешней среде?
% 12. Какое множество альтернатив системы (дискретное, непрерывное, дискретно–непрерывное) Вы рассматриваете?
% 13. Назовите основные структуры и параметры, характеризующие рассматриваемое Вами множество альтернатив системы.
% 14. При каких ограничениях на значения непрерывных параметров системы решается задача?
% 15. Составьте функциональную схему и опишите функциональные связи между подсистемами в составе системы, а также между подсистемами и внешней средой, которые Вы предполагаете учитывать при разработке математической модели системы.
% 16. Какую систему (или системы) Вы готовы рассматривать как актуальный прототип по отношению к системе, рассматриваемой в Вашей работе?
% 17. Какую задачу – анализа или синтеза системы – Вы решаете? Как решаемая задача связана с целью Вашего исследования?
% 18. Укажите тип математической модели (аналитическая, основанная на использовании физических законов и/или теории, имитационная, эмпирическая), которую Вы будете использовать для решения задачи анализа системы.
% 19. Охарактеризуйте кратко особенности разрабатываемой Вами модели системы.
% 20. В каком состоянии находится разработка модели в настоящее время?
% 21. В какой среде программирования Вы реализуете модель системы?
% 22. Если в работе решается задача синтеза системы, то какой алгоритм оптимизации альтернативы системы Вы используете или предполагаете использовать?
% 23. Если при синтезе системы рассматриваются несколько показателей ее эффективности, то как решается задача оптимизации системы по векторному критерию?
% 24. Какие новые научные и/или практические результаты Вы уже получили (или предполагаете получить) в Вашем исследовании?
% 25. Есть ли у Вас публикации по работе? Выступали ли Вы на научных конференциях по теме Вашей диссертации? Перечислите публикации, укажите место выступления (выступлений).

\begin{document}

    \begin{center}
    \Large
    \textbf{Реферат}
    \end{center}

    \begin{enumerate}
        \item \textit{Какую систему Вы изучаете в Вашей диссертационной работе? Что является объектом исследования.}

        % 1. Какую систему Вы изучаете в Вашей диссертационной работе? Что является объектом исследования.

Объектом моего диссертационного исследования являются плагинные системы. В своем диссертационном исследовании я определяю плагинную систему как способ реализации приложения. Возможности приложения при таком способе организации ограничиваются объмом возможностей установленных расширений приложения - плагинов.

Пример плагинной системы - Eclipse IDE. Это выполненная в виде плагинной системы среда разработки ПО. Система образуется за счет того, что плагины могут взаимодействовать друг с другом и раширять уже имеющийся функционал других плагинов. 


        \item \textit{Каково назначение Вашей системы?}

        % 2. Каково назначение Вашей системы?

В моей диссертационной работе плагинные системы рассматриваются с позиции их использования в качестве среды разработки ПО.

Тогда назначением плагинной системы является является формирование такой инструментальной среды, которая позволяла бы оператору применять свои навыки и умения в рамках имеющейся квалификации для решения бизнес задач в заданной предметной области.

        \item \textit{Какова цель исследования системы? Где она может быть использована?}

        % 3. Какова цель исследования системы? Где она может быть использована?



        \item \textit{Частью какой надсистемы является изучаемая система?}

        % 4. Частью какой надсистемы является изучаемая система?
Для плагинной системы как среды разработки ПО надсистемой является процесс разработки ПО.

Процесс разработки ПО включает следующие процессы:
\begin{enumerate}
    \item процесс проектирования;
    \item процесс разработки требований;
    \item процесс кодирования;
    \item процесс интеграции.
\end{enumerate}


        \item \textit{Из каких подсистем состоит система?}

        % 5. Из каких подсистем состоит система?
Плагинная система состоит из плагинов, состав которых определяет ее функциональные возможности.

Для использования плагинной системы как инструментальной среды разработки ПО считаю, что она должна включать такие плагины как:
\begin{itemize}
    \item плагин, предоставляющий редактор с подсветкой синтаксиса и возможностями автодополнения для использумого в разрабатываемом проекте языка программирования;
    \item плагин навигации и поиска компонетов в проекте разрабатываемого ПО;
    \item плагин управления составом подключаемых компиляторов и правилами сборки проекта разрабатываемого ПО;
    \item плагин взаимодействия с системой контроля версий (СКВ) и др.
\end{itemize}

        \item \textit{Какие задачи решают подсистемы в составе Вашей системы?}

        % 6. Какие задачи решают подсистемы в составе Вашей системы?
Плагин, предоставляющий редактор с подсветкой синтаксиса и возможностями автодополнения для использумого в разрабатываемом проекте языка программирования решает следующие задачи:
\begin{itemize}
    \item обеспечение оператора текстовым редактором;
    \item сокращение времени написания исходного кода за счет предоставления оператору вариантов автодополнения;
    \item уменьшение когнитивной сложности за счет подсветки ключевых слов в тексте исходного кода.
\end{itemize}

Плагин навигации и поиска компонетов в проекте разрабатываемого ПО решает следующие задачи:
\begin{itemize}
    \item уменьшение когнитивной сложности навигации по структуре проекта за счет отображения структуры проекта, например, в виде дерева;
    \item уменьшение когнитивной сложности оценки объема функционала реализованного в файле исходного кода за счет предоставления списка описанных функций или классов;
    \item сокращение фремени поиска файла исходного кода проекте разрабатываемого ПО по имени или содержимому.
\end{itemize}

Плагин управления составом подключаемых компиляторов и правилами сборки проекта разрабатываемого ПО решает следующие задачи:
\begin{itemize}
    \item уменьшение сложности управления составом ключей компиляторов;
    \item информирование оператора об используемом в процессе сборки компилятора;
    \item уменьшение когнитивной сложности анализа результата работы компилятора при возникновении ошибок компиляции.
\end{itemize}

Плагин взаимодействия с СКВ решает следующие задачи:
\begin{itemize}
    \item обеспечение пользовательским интерфейсом для взаимодействия с СКВ;
    \item информарирование оператора о применяемой базовой версии для внесения изменений в проект разрабатываемого ПО;
    \item автоматизация внесения изменений в проект разрабатываемого ПО в соответствии с действующей на проекте дисциплиной процесса управления конфигурацией.
\end{itemize}

        \item \textit{Сформулируйте кратко сценарий функционирования системы.}

        % 7. Сформулируйте кратко сценарий функционирования системы.
Для среды разработки ПО выделяю следующий сценарий функционирования:
\begin{enumerate}
    \item установка на базовую версию проекта разрабатываемого ПО;
    \item внесение изменений в один или несколько файлов исходного кода;
    \item выполнение формирования файлов исполняемого объектоного кода и сборка исполняемого образа с формированием файлов компонентов параметрических данных;
    \item загрузка исполняемого образа и файлов компонентов параметрических данных на целевой вычислитель;
    \item фиксация внесенных изменений в соответствии с дисциплиной процесса управления конфигурацией.
\end{enumerate}

        \item \textit{Какие факторы внешней среды Вы учитываете при анализе функционирования системы?}

        % 8. Какие факторы внешней среды Вы учитываете при анализе функционирования системы?

При анализе функционирования системы учитываются следующие факторы внешней среды:
\begin{itemize}
    \item программно-аппаратные характеристики вычислительной системы, в которой выполняется работа среды разработки ПО:
        \begin{enumerate}
            \item операционная система;
            \item тип, количество ядер и тактовая частота процессора;
            \item объем постоянной и оперативной памяти.
        \end{enumerate}
    \item частота и количество воздействий оператора на элементы управления среды разработки ПО;
    \item факторы сетевого обмена:
        \begin{enumerate}
            \item пропускная способность сети;
            \item количество абонентов;
            \item используемые протоколы сетевого обмена.
        \end{enumerate}
\end{itemize}

        \item \textit{Какой информацией о факторах внешней среды Вы располагаете (детерминированные, случайные, интервально неопределенные, активное противодействие «противника» или конкурента)?}

        % 9. Какой информацией о факторах внешней среды Вы располагаете (детерминированные, случайные, интервально неопределенные, активное противодействие «противника» или конкурента)?
К детерминированным факторам внешней среды отношу:
\begin{enumerate}
    \item количество одновременно взаимодействующих со средой операторов - оператор всегда один;
    \item количество элементов управления, через которое оператор взаимодействует со средой разработки - исключено взаимодействие одновременно с несколькими элементами управления;
    \item среда разработки осуществляет свою работу под управлением одной операционной системы.
\end{enumerate}

К случайным факторам внешней среды отношу:
\begin{enumerate}
    \item на какой следующий элемент управления будет осуществлено воздействие оператора;
    \item количество ресурсов вычислительной системы, доступных для работы:
    \begin{itemize}
        \item количество процессного времени и объем оперативной памяти влияет на быстродействие среды разработки;
        \item политика безопасности определяет характер взаимодействия с ресурсами операционной системы, например, может не выполняться операция записи в файл журнала.
    \end{itemize}
\end{enumerate}

Конкурентная борьбся моей проектируемой системы от существующих аналогов на рынке осуществляется по критериям:
\begin{enumerate}
    \item объем функционала в его среде разработке;
    \item стоимость поставляемого функционала;
    \item время формирования поставки.
\end{enumerate}

        \item \textit{Какие показатели эффективности системы в целом и ее подсистем Вы рассматриваете?}

        % 10. Какие показатели эффективности системы в целом и ее подсистем Вы рассматриваете?

Считаю, что эффективность системы обратно пропорциональна числу невостребованного заказчику функционала в поставке программного решения.

Так, выделяю коэффициент эффективности поставки:

\begin{center}
    $k = v^* - (v_{u} / v^*)$
\end{center}

где:

$k$ - коэффициент эффективности;

$v^*$ - количество требований к ПО, реализованных в рамках поставки программного решения;

$v_u$ - количество требований к ПО, реализованных в рамках поставки программного решения и невостребованных для заказчика.

Как видно из формулы коэффициент эффективности максимален и равен 1 тогда, и только тогда, когда число $v_{u}$ равно 0, т.е. в поставке отсутствует реализация невостребованных для заказчика требований. И минимален, равен 0 тогда, и только тогда, когда число $v_{u}$ равно $v^*$, т.е. все требования, реализованные в рамках поставки, невостребованы для заказчика.

\textit{Примечание: считается, что поставка содержит реализацию всех востребованных для заказчика требований. Таким образом $k$ всегда больше 0.}

К показателям эффективности плагинов отношу:
\begin{enumerate}
    \item время отклика компонента на воздействие оператора на элемент управления, относящегося к нему.
    \item отказоустойчивость компонента. Например, если в программном решении не предусмотрен обработчик ситуации при которой невозможно осуществить взаимодействие с ресурсом вычислительной системы, то невозможно гарантировать дальнейшую корректную работу всего программного решения.
\end{enumerate}

        \item \textit{Как в этих показателях учитывается информация о внешней среде?}

        % 11. Как в этих показателях учитывается информация о внешней среде?

Информация о внешней среде учитываются в каждом из обозначенных показателях эффективности.

Показатель эффективности к системе вцелом относится к аспектам конкуррентной борьбы. Показатели эффективности подсистем относятся к аспектам случайных факторов.

    \end{enumerate}


\end{document}