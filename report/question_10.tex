% 10. Какие показатели эффективности системы в целом и ее подсистем Вы рассматриваете?

Считаю, что эффективность системы обратно пропорциональна числу невостребованного заказчику функционала в поставке программного решения.

Так, выделяю коэффициент эффективности поставки:

\begin{center}
    $k = v^* - (v_{u} / v^*)$
\end{center}

где:

$k$ - коэффициент эффективности;

$v^*$ - количество требований к ПО, реализованных в рамках поставки программного решения;

$v_u$ - количество требований к ПО, реализованных в рамках поставки программного решения и невостребованных для заказчика.

Как видно из формулы коэффициент эффективности максимален и равен 1 тогда, и только тогда, когда число $v_{u}$ равно 0, т.е. в поставке отсутствует реализация невостребованных для заказчика требований. И минимален, равен 0 тогда, и только тогда, когда число $v_{u}$ равно $v^*$, т.е. все требования, реализованные в рамках поставки, невостребованы для заказчика.

\textit{Примечание: считается, что поставка содержит реализацию всех востребованных для заказчика требований. Таким образом $k$ всегда больше 0.}

К показателям эффективности плагинов отношу:
\begin{enumerate}
    \item время отклика компонента на воздействие оператора на элемент управления, относящегося к нему.
    \item отказоустойчивость компонента. Например, если в программном решении не предусмотрен обработчик ситуации при которой невозможно осуществить взаимодействие с ресурсом вычислительной системы, то невозможно гарантировать дальнейшую корректную работу всего программного решения.
\end{enumerate}