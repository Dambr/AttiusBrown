% 6. Какие задачи решают подсистемы в составе Вашей системы?
Плагин, предоставляющий редактор с подсветкой синтаксиса и возможностями автодополнения для использумого в разрабатываемом проекте языка программирования решает следующие задачи:
\begin{itemize}
    \item обеспечение оператора текстовым редактором;
    \item сокращение времени написания исходного кода за счет предоставления оператору вариантов автодополнения;
    \item уменьшение когнитивной сложности за счет подсветки ключевых слов в тексте исходного кода.
\end{itemize}

Плагин навигации и поиска компонетов в проекте разрабатываемого ПО решает следующие задачи:
\begin{itemize}
    \item уменьшение когнитивной сложности навигации по структуре проекта за счет отображения структуры проекта, например, в виде дерева;
    \item уменьшение когнитивной сложности оценки объема функционала реализованного в файле исходного кода за счет предоставления списка описанных функций или классов;
    \item сокращение фремени поиска файла исходного кода проекте разрабатываемого ПО по имени или содержимому.
\end{itemize}

Плагин управления составом подключаемых компиляторов и правилами сборки проекта разрабатываемого ПО решает следующие задачи:
\begin{itemize}
    \item уменьшение сложности управления составом ключей компиляторов;
    \item информирование оператора об используемом в процессе сборки компилятора;
    \item уменьшение когнитивной сложности анализа результата работы компилятора при возникновении ошибок компиляции.
\end{itemize}

Плагин взаимодействия с СКВ решает следующие задачи:
\begin{itemize}
    \item обеспечение пользовательским интерфейсом для взаимодействия с СКВ;
    \item информарирование оператора о применяемой базовой версии для внесения изменений в проект разрабатываемого ПО;
    \item автоматизация внесения изменений в проект разрабатываемого ПО в соответствии с действующей на проекте дисциплиной процесса управления конфигурацией.
\end{itemize}