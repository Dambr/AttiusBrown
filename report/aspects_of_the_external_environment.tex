% Аспекты внешней среды
% 8. Какие факторы внешней среды Вы учитываете при анализе функционирования системы?
% 9. Какой информацией о факторах внешней среды Вы располагаете (детерминированные, случайные, интервально неопределенные, активное противодействие «противника» или конкурента)?
% 10. Какие показатели эффективности системы в целом и ее подсистем Вы рассматриваете?
% 11. Как в этих показателях учитывается информация о внешней среде?

Среди аспектов внешней среды системы я выделил изначально сформированный объем требований к системе и требования заказчика, которые могут изменяться от поставки к поставке. От потребных заказчику в рамках поставки объема функционала зависит цена послепродажного обслуживания ПО.

Например, заказчик заказал 10 требований, а поставлен был функционал, реализующий 15. 5 требований он не оплатил, но осуществлять их поддержку все равно необходимо. В данном случае коэффициент простоя равен $5 / 15 = 1/3$. Почти 30\% работы персонала по сопровождению заказчиком не оплачена.

С другой стороны существуют затраты на разработку и поддержку IT продукта. Чем больше в нем сущностей, тем дороже управление его конфигурацией, сложение отношения между сущностями и усложняется поиск и отладка неисправностей. Не говоря уже о когнитивном усложнении восприятии проекта и, как следствие, повышение требований к квалификации разработчиков, что в свою очередь тоже может быть переведено в денежный эквивалент.

Подытоживая, от этих денежных показателей зависит рентабельность IT продукта, а значит и конкурентоспособность бизнеса.