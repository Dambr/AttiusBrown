В современном мире важную роль в жизни общества занимают информационные технологии. В частности информационные технологии охватывают программные средства, применяемые для цифровых устройств, которые человек использует в повседневной жизни.

Отдельно выделяют класс программных средств - плагинные системы. Это способ организации приложения таким образом, что его конечный объем функционала характеризуется количеством установленных в него расширений - плагинов.

Каждый отдельный плагин включает в себя конечное множество функционала, который в свою очередь основан на требованиях. Именно характер реализованных требований, а так же их объем потребен заказчику. Все остальное - плагины, их взаимосвязи, язык программирования, на котором они реализованы, библиотеки, которые задействованы - все это скрыто от заказчика и зачастую его не интересует. Заказчику интересно, какие свои бытовые или бизнес потребности он сможет удовлетворить от применения программного средства.

На стадии проектирования приложения зачастую неизвестно, какие требования будут востребованы у заказчиков и определить структуру приложения невозможно. Кроме того, для разных заказчиков потребен разный функционал. При необходимости формирования поставки, включающей требуемый объем функционала, поставщик зачастую включает и тот функционал, который не востребован и не оплачен заказчиком, но без которого не быть поставлен востребованный. Это связано с существованием зависимостей у реализации.

Моя диссертационная работа посвящена поиску оптимальной структуры приложения, которая бы с одной стороны позволяла формировать поставки с минимальным числом невостребованного у заказчика функционала, а с другой сдерживала неконтролируемый рост кодовой базы, тем самым, сдерживала стоимость разработки и сопровождения проекта.