% Заключение
% 24. Какие новые научные и/или практические результаты Вы уже получили (или предполагаете получить) в Вашем исследовании?
% 25. Есть ли у Вас публикации по работе? Выступали ли Вы на научных конференциях по теме Вашей диссертации? Перечислите публикации, укажите место выступления (выступлений).

Полученные мною результаты применяются для различных приложений и решений выполненных как набор плагинов для интеграции в соответствующие им плагинные системы.

Так, на конференции в МГТУ им. Баумана я в своем докладе «Исследование правил построения конфигуратора ARINC 653 спецификации в IDE Eclipse» показал результаты применения сформулированной модели для решения задачи путем разбиения файлов исходного кода по плагинам по частотам вызова функций. В качестве результата продемонстрировал снижение невостребованного функционала до 10\% при различных запросах состава требований.

На конференции в Воронеже в ВУНЦ ВВС «ВВА имени профессора Н.Е. Жуковского и Ю.А. Гагарина» в своем докладе «Реализация инструментального средства конфигурирования компонентов БРЭО» показал актуальность применения плагинных систем для решения задач в авиационной сфере.

Подготовил выступление на конференции, организуемой Союзом Машиностроителей России. В качестве материала выступления мною предоставлено средство интеллектуального конфигурирования системы определения состояния воздушного судна. В этой работе я сформировал среду конфигурирования, которая независимо подключается к базе данных и заполняет ее конфигурационными параметрами и их значениями с целью дальнейшей их обработки непосредственно в самой системе.

Подготовлена публикация ВАК, в которой описан алгоритм разрешения циклических зависимостей графовой модели трассируемости требований к ПО на файлы исходного кода. Работа передана в редакцию ИПМ им. М.В.Келдыша РАН.
