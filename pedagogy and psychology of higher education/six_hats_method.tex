Автор этой техники — британский психолог Эдвард де Боно. По его мнению, люди привыкают воспринимать реальность с одной и той же точки зрения, то есть <<носят одну и ту же шляпу>>. Из-за этого креативности в мышлении становится меньше, а стереотипов — больше. Чтобы начать мыслить по-другому, нужно <<примерить другую шляпу>>. Мозговой штурм в этой технике можно строить по нескольким сценариям. Например, свой цвет шляпы может быть у каждого участника команды, или все <<примеряют>> одинаковые шляпы и одновременно меняют их. Вот как это может выглядеть.

\textbf{Белая шляпа:} факты, объективность и прояснение информации. Например, команда разработчиков озвучивает факт о том, как идёт процесс написания кода: <<За три дня мы сделали 20\% работы, а планировали выполнить 40\%>>.

\textbf{Красная шляпа:} чувства и эмоции. Тот, кто <<носит>> эту шляпу, может сказать о любых чувствах, которые вызывают у него факты, например: <<В самом начале мы переживали, так как думали, что не справимся>>, <<Мы молодцы, потому что смогли собраться и сделать максимум в сложившейся ситуации>>.

\textbf{Чёрная шляпа:} критика. Участник в этой шляпе может оценивать факты и объективно их критиковать, например: <<Если бы у всей команды был примерно одинаковый опыт, мы бы сделали запланированное. А так нам пришлось потратить время на разбор задачи>>.

\textbf{Жёлтая шляпа:} достоинства и позитивные моменты. На этом этапе мозгового штурма нужно найти и озвучить положительные моменты, даже если их немного: <<Коллега работал на других проектах, но у него получилось быстро переключиться на наши задачи>>.

\textbf{Зелёная шляпа:} творчество и креатив. После того как участники озвучили плюсы и минусы ситуации и поделились эмоциями, нужно решить, как улучшить работу команды на следующем этапе, например: <<Сразу выясняем, какими навыками обладают новые участники>>, <<Распределяем задачи в зависимости от опыта специалистов>>.

\textbf{Синяя шляпа:} подведение итогов. Это заключительный этап мозговой атаки, на котором нужно систематизировать идеи и сформулировать результат встречи, например: <<Мы не выполнили план по задаче, потому что в начале возникли сложности с менее опытными членами команды. Чтобы на следующем этапе такого не было, решили сразу выяснять уровень опыта новых участников и в зависимости от этого распределять задачи>>.
