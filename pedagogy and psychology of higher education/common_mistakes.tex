Чтобы мозговой штурм оправдал ожидания участников, нужно помнить о правилах и учитывать ошибки, которые допускают чаще всего:

\begin{enumerate}
\item Размытая тема. Все участники должны чётко понимать цель встречи и <<штормить>> только по ней, не отклоняясь в стороны.

\item Слишком много/мало участников. Если в собрании участвует 3–4 человека, есть риск, что <<банк идей>> быстро иссякнет и цель мозгового штурма не будет достигнута. Слишком много людей — тоже плохо. Тогда генерировать идеи будут самые активные члены команды, а остальные — отмалчиваться.

\item Очерёдность выступлений. Если в команде участники с разным профессиональным бэкграундом, стоит начинать с <<младших>>, например, сначала свои идеи высказывают младшие специалисты, затем более опытные и только в конце — руководитель коллектива. Так старшие не будут давить своим авторитетом, а подчинённые не будут испытывать эффекта согласия с руководством.

\item <<Одинаковость>> участников. Чтобы после мозгового штурма получить много разных идей, в собрании должны участвовать люди разных профессий, опыта, взглядов на жизнь. Так получится дополнять идеи друг друга, а шансы найти подходящее решение вырастут.

\item Не все идеи зафиксированы. Когда предложений слишком много, а правила и принципы мозгового штурма не соблюдаются, можно упустить действительно важные и необычные решения.
\end{enumerate}
